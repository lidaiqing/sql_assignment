\documentclass{article}
\usepackage{fullpage}
\usepackage[normalem]{ulem}
\usepackage{amstext}
\usepackage{amsmath}
\newcommand{\var}[1]{\mathit{#1}}
\setlength{\parskip}{6pt}

\begin{document}

\noindent
University of Toronto\\
{\sc csc}343, Fall 2016\\[10pt]
{\LARGE\bf Assignment 1: Daiqing Li 1000795357} \\[10pt]

\noindent
Unary operators on relations:
\begin{itemize}
\item $\Pi_{x, y, z} (R)$
\item $\sigma_{condition} (R) $
\item $\rho_{New} (R) $
\item $\rho_{New(a, b, c)} (R) $
\end{itemize}
Binary operators on relations:
\begin{itemize}
\item $R \times S$
\item $R \bowtie S$
\item $R \bowtie_{condition} S$
\item $R \cup S$
\item $R \cap S$
\item $R - S$
\end{itemize}
Logical operators:
\begin{itemize}
\item $\vee$
\item $\wedge$
\item $\neg$
\end{itemize}
Assignment:
\begin{itemize}
\item $New(a, b, c) := R$
\end{itemize}
Stacked subscripts:
\begin{itemize}
\item
$\sigma_{\substack{this.something > that.something \wedge \\ this.otherthing \leq that.otherthing}}$
\end{itemize}

\noindent
Below is the text of the assignment questions; we suggest you include it in your solution.
We have also included a nonsense example of how a query might look in LaTeX.  
We used \verb|\var| in a couple of places to show what that looks like.  
If you leave it out, most of the time the algebra looks okay, but certain words,
{\it e.g.}, ``Offer" look horrific without it.

The characters ``\verb|\\|" create a line break and ``[5pt]" puts in 
five points of extra vertical space.  The algebra is easier to read with extra
vertical space.
We chose ``--" to indicate comments, and added less vertical space between comments
and the algebra they pertain to than between steps in the algebra.
This helps the comments visually stick to the algebra.


%----------------------------------------------------------------------------------------------------------------------
\section*{Part 1: Queries}

\begin{enumerate}

\item   % ----------
Find all patients who (a) have had more than 2 different doctors write them a prescription,
and (b) have had a narcotic prescribed to them by every doctor who has written them a prescription.
A narcotic is a drug whose schedule is ``narcotics".
Report the patient's OHIP number. \\

{\large %This increase in font size makes the subscripts much more readable.
-- sID has a grade of at least 85. \\[5pt]
$
HaveHighGrade(\var{sID}) := 
	\Pi_{sID} 
	\sigma_{grade \ge 85} 
	Took \\[10pt]
$
-- sID passed a course taught by Atwood. \\[5pt]
$
PassedAtwood(\var{sID}) := 
	\Pi_{\var{sID}} 
	\sigma_{instructor := ``Atwood" \wedge grade \ge 50} 
	(Took \bowtie \var{Offering}) 
	\\[10pt]
$
-- sID got 100 at least twice. \\[5pt]
$
AtLeastTwice(\var{sID}) := \\[5pt]  %This RA statement is too long, so we break it into two lines.
	\hspace*{1cm}  % This command creates an indentation
	\Pi_{T1.\var{sID}} 
	\sigma_{
		T1.\var{oID} \neq T2.\var{oID} \wedge 
		T1.\var{sID} = T2.\var{sID} \wedge 
		T1.grade = 100 \wedge T2.grade = 100}
	[ (\rho_{T1}Took) \times (\rho_{T2}Took) ] \\[5pt]
$
} % End of font size increase.
{\large
$
HaveAtLeastThreeDoctor(OHIP):= \\[5pt]
	\hspace*{1cm}
	\Pi_{patient}
	\sigma_{(P1.\var{doctor} < P2.\var{doctor} < p3.\var{doctor})\wedge(P1.patient=P2.patient=P3.patient)}
	[(\rho_{P1}Prescription) \times (\rho_{P2}Prescription) \times (\rho_{P3}prescription)] \\[5pt]
NonNarcotics(OHIP):= \\[5pt]
	\hspace*{1cm}
	\Pi_{patient}
	\sigma_{schedule \neq ``narcotic"}
	(Prescription \bowtie_{Prescription.\var{drug} = Product.\var{DIN}} Product) \\[5pt]  
ANS(OHIP):= HaveAtLeastThreeDoctor(OHIP) - NonNarcotics(OHIP) \\[5pt]  
$
}
\item   % ----------
Find every prescription from 2016 that has never been filled.
Report the patient's OHIP number, the prescription ID, prescription date, and drug. \\

{\large
$
AllPrescriptionSince2016(OHIP,RxID,date,drug):= \\[5pt]
	\hspace*{1cm}
	\Pi_{patient, RxID, date, drug}
	\sigma_{date \ge 2016}
	(Prescription) \\[5pt]
AllPrescriptionFilledSince2016(OHIP,RxID,date,drug):= \\[5pt]
	\hspace*{1cm}
	\Pi_{patient, RxID, Prescription.\var{date}, drug}
	\sigma_{Prescription.\var{date} \ge 2016}
	(Prescription \bowtie Filled) \\[5pt]
ANS(OHIP,RxID,date,drug):= \\[5pt]
 	\hspace*{1cm}
 	AllPrescriptionSince2016(OHIP,RxID,date,drug) - \\
 	\hspace*{2cm}
 	AllPrescriptionFilledSince2016(OHIP,RxID,date,drug) \\[5pt]  
$
}
\item   % ---------- 
Find the pharmacist who has trained the most people.
Report the pharmacist's OCP number and name.


\item   % ----------
The ``narcotics prescription period" of a doctor for a patient is the time from the first
prescription for narcotics from that doctor for that patient to the most recent one.
(It would be zero if that doctor wrote only one prescription for narcotics for that patient.)
Find all patients who have had narcotics prescribed by two or more doctors,
and for whom the narcotics prescription periods never overlap.
In other words, if they had narcotics prescribed by $n$ different doctors,
$$[start_1..end_1] < [start_2..end_2] < ... < [start_n ..end_n]$$
where $start_i$ and $end_i$ are the start and end of the 
narcotics prescription period of doctor $i$ for that patient.
Notice that we have written strictly less than. 
This means that if $end_i = start_{i+1}$ we do not consider that the periods overlap.
Report the patient's OHIP number.


\item   % ----------
Find all pharmacists who have never filled a prescription for a drug product whose active ingredient is ``codeine".
Report their OCP number and every schedule for which they {\it have} filled a prescription.
Put the information into a relation with attributes ``OCP" and ``schedule". \\

{\large
$
AllFilled(OCP, schedule):= \\[5pt]
	\hspace*{1cm}
	\Pi_{pharmacist, schedule}
	(Prescription \bowtie Filled \bowtie_{Prescription.\var{drug} = Product.\var{DIN}} Product) \\[5pt]
DrugCodeline(drug):= \\[5pt]
	\hspace*{1cm}
	\Pi_{DIN}
	(Product \bowtie_{Product.DIN=ActiveIngredient.DIN \wedge ActiveIngredient.name=``codeline"} \\
	\hspace*{4cm}	
	ActiveIngredient) \\[5pt]
FilledCodeline(OCP, schedule):= \\[5pt]
	\hspace*{1cm}
	\Pi_{pharmacist, schedule}
	(Prescription \bowtie Filled \bowtie DrugCodeline \bowtie_{Prescription.\var{drug} = Product.\var{DIN}} Product) \\[5pt]
ANS(OCP, schedule):= \\[5pt]
 	\hspace*{1cm}
 	AllFilled(OCP, schedule) - FilledCodeline(OCP, schedule) \\[5pt]
$
}


\item   % ----------
Let�s say a minor trainer is a pharmacist who has trained no more than two people.
(They may have trained none.) 
Find all pharmacists who have trained 2 or more minor trainors. 
(They may have trained other pharmacists who were not minor trainors.) 
Report the pharmacist's OCP number. \\


{\large
$
TrainedAtLeastThree(OCP):= \\[5pt]
	\hspace*{1cm}
	\Pi_{P1}
	\sigma_{
		T1.\var{P1} < T2.\var{P1} < T3.\var{P1} \wedge 
		T1.\var{P2} = T2.\var{P2} = T3.\var{p2} }
	[ (\rho_{T1}TrainedUnder) \times (\rho_{T2}TrainedUnder) \times (\rho_{T3}TrainedUnder) ] \\[5pt]	
MinorTrainer(OCP):= \\[5pt]
	\hspace*{1cm}
	TrainedAtLeastThree - 
	\Pi_{OCP}
	(Pharmacist) \\[5pt]
TrainMinorAtLeastOne(P2):= \\[5pt]
	\hspace*{1cm}
	\pi_{P2}
	(TrainedUnder \bowtie_{MinorTrainer.OCP=TrainedUnder.P1} MinorTrainer) \\[5pt]
ANS(OCP):= \\[5pt]
 	\hspace*{1cm}
 	\Pi_{P2}
	\sigma_{
		T1.\var{P1} < T2.\var{P1} < T3.\var{P1} \wedge 
		T1.\var{P2} = T2.\var{P2} = T3.\var{p2}}
	[ (\rho_{T1}TrainMinorAtLeastOne) \times \\
	(\rho_{T2}TrainMinorAtLeastOne) \times (\rho_{T3}TrainMinorAtLeastOne) ] \\[5pt]
$
}


\item   % ----------
Find the most junior pharmacist:
the pharmacist whose first time filling a prescription has the latest date.
Report the pharmacist's OCP number, the prescription ID for the first prescription they filled,
the date on which it was written, and the date on which it was filled. \\

{\large
$
All(OCP, RxID, writeDate, filldate):= \\[5pt]
	\hspace*{1cm}
	\Pi_{pharmacist, RxID, writedate, filldate}  
	((\rho_{date->filldate}Filled) \bowtie (\rho_{date->writedate}Prescription) \\[5pt]	
NotLatest(OCP, RxID, writeDate, filldate):= \\[5pt]
	\hspace*{1cm}
	\Pi_{A1.pharmacist, A1.RxID, A1.writedate, A1.filldate}
	\sigma_{A1.date < A2.date}
	[(\rho_{A1}ALL) \times (\rho_{A2}ALL)] \\[5pt]
ANS(OCP, RxID, writeDate, filldate):= \\[5pt]
 	\hspace*{1cm}
 	\Pi_{pharmacist, RxID, writedate, filldate}
	(ALL - NotLatest)\\[5pt]
$
}


\item   % ----------
Find every patient who has had a prescription for a homeopathic drug product filled,
that is, a product whose schedule is ``homeopathic",
but has never had a prescription filled for a drug product with any other schedule.


\item   % ----------
Find all patients who have had at least two prescriptions for narcotics that have a
single active ingredient, whose units are mg,
and for whom the dosage of the ingredient in these prescriptions never decreased
from one prescription to the next.
Report their OHIP number.


\item   % ----------
Report the OCP number of the pharmacist who has had the greatest number of other pharmacists train under him or her.


\item   % ----------
For each pharmacist who has trained anyone,
report their OCP number, the OCP number of the first person to complete training under them, and the OCP number of the last person to complete training under them.
Your resulting relation should have three attributes: ``OCP", ``first" and ``last".


\item   % ----------
Find all people who have, at least twice, had more than one prescription filled in a year, but haven't had one filled since 2014. 
Report the person's OHIP number and the last date on which they had
a prescription filled.

\end{enumerate}



%----------------------------------------------------------------------------------------------------------------------
\section*{Part 2: Additional Integrity Constraints}

\begin{enumerate}

\item   % ----------
A pharmacist can only train under someone who registered with the Ontario College of Physicians before they did.


\item   % ----------
A doctor can't prescribe a controlled substance (a product with schedule ``narcotics")
until after they have prescribed three different over-the-counter drug products (products with schedule ``OTC").

\end{enumerate}

\end{document}



